

% This is a small sample LaTeX input file (Version of 10 April 1994)
%
% Use this file as a model for making your own LaTeX input file.
% Everything to the right of a  %  is a remark to you and is ignored by LaTeX.

% The Local Guide tells how to run LaTeX.

% WARNING!  Do not type any of the following 10 characters except as directed:
%                &   $   #   %   _   {   }   ^   ~   \

%%%%%%%%%%%%%%%%%%%%%%% file typeinst.tex %%%%%%%%%%%%%%%%%%%%%%%%%
%
% This is the LaTeX source for the instructions to authors using
% the LaTeX document class 'llncs.cls' for contributions to
% the Lecture Notes in Computer Sciences series.
% http://www.springer.com/lncs       Springer Heidelberg 2006/05/04
%
% It may be used as a template for your own input - copy it
% to a new file with a new name and use it as the basis
% for your article.
%
% NB: the document class 'llncs' has its own and detailed documentation, see
% ftp://ftp.springer.de/data/pubftp/pub/tex/latex/llncs/latex2e/llncsdoc.pdf
%
%%%%%%%%%%%%%%%%%%%%%%%%%%%%%%%%%%%%%%%%%%%%%%%%%%%%%%%%%%%%%%%%%%%

\documentclass[conference]{IEEEtran}

%% INFOCOM 2011 addition:
\makeatletter
\def\ps@headings{%
	\def\@oddhead{\mbox{}\scriptsize\rightmark \hfil \thepage}%
	\def\@evenhead{\scriptsize\thepage \hfil \leftmark\mbox{}}%
	\def\@oddfoot{}%
	\def\@evenfoot{}}
\makeatother
\pagestyle{headings}
%\usepackage{latex8}
\usepackage{setspace}
\usepackage{times}
 
\usepackage{pgf}
\usepackage{tikz}
\usetikzlibrary{arrows,automata}
\usepackage[latin1]{inputenc}
\usetikzlibrary{arrows}
\usepackage{epsfig,xspace}
\usepackage{url}
\usepackage{verbatim}
\usepackage{amsmath}
\usepackage{listings}
\usepackage{verbatim}
\usepackage{varwidth}
\usepackage{graphicx}
\usepackage{mathtools}
%\usepackage{fix2col}
\usepackage{multirow}

\usepackage{algorithm}
\usepackage{algorithm}
\usepackage{algpseudocode}
%\usepackage[dvips]{graphicx}
%\usepackage{mss}
%\usepackage{subfigure}
\usepackage{subfigure}

%\usepackage{nopageno}
%\usepackage{fix2col}

\IEEEoverridecommandlockouts
\lstset{
	basicstyle=\footnotesize,
	columns=flexible,
	morecomment=[s]{/*}{*/}}

\newtheorem{Theorem}{Theorem}[section]
\newtheorem{lemma}[Theorem]{Lemma}
\newtheorem{proposition}[Theorem]{Proposition}
\newtheorem{corollary}[Theorem]{Corollary}

\newcommand{\remove}[1]{ }
\newcommand{\eg}{\emph{e.g.}}
\newcommand{\ie}{\emph{i.e.}}

\newenvironment{proof}[1][Proof:]{\begin{trivlist}
		\item[\hskip \labelsep {\bfseries #1}]}{\end{trivlist}}


\newenvironment{definition}[1][Definition]{\begin{trivlist}
		\item[\hskip \labelsep {\bfseries #1}]}{\end{trivlist}}
\newenvironment{example}[1][Example]{\begin{trivlist}
		\item[\hskip \labelsep {\bfseries #1}]}{\end{trivlist}}
\newenvironment{remark}[1][Remark]{\begin{trivlist}
		\item[\hskip \labelsep {\bfseries #1}]}{\end{trivlist}}

\renewcommand{\algorithmicrequire}{\textbf{Input:}}
\renewcommand{\algorithmicensure}{\textbf{Output:}}

\newcommand{\qed}{\nobreak \ifvmode \relax \else
	\ifdim\lastskip<1.5em \hskip-\lastskip
	\hskip1.5em plus0em minus0.5em \fi \nobreak
	\vrule height0.6em width0.5em depth0.0em\fi}

\def\sqzhuge{\vspace{-14pt}}
\def\sqzsec{\vspace{-10pt}}
\def\sqzsmall{\vspace{-8pt}}
\def\sqztiny{\vspace{-4pt}}
\remove{
	\usepackage{vmargin}
	\setpapersize{USletter}
	\setmarginsrb{2.54cm}{2.54cm}{2.54cm}{1.5cm}{0pt}{0mm}{0pt}{8mm}
	\setcounter{secnumdepth}{3}
}

\begin{document}
 

%\title{Infrastructure Notes} 
%\vspace{-0.4in}
%\maketitle
 
 
\title{Privacy-aware Traffic Measurement through Vehicle-to-Infrastructure Communication}


\maketitle
\vspace{-1in}
\begin{abstract}
	Persistent traffic measurement is an emerging topic that has many applications in modern transportation system. With the advances in technologies of vehicle networks, modern transportation system is able to provide sophisticated traffic measurements with the help of Vehicle-to-Infrastructure (V2I) communications. With recent progress in estimation of unions and intersections, we propose a new estimator for persistent point traffic measurement and persistent point-to-point traffic measurement based on Flajolet-Martin sketch used in HyperLogLog. We evaluate our estimation methods using simulations based on both real transportation traffic data and synthetic data. We compare our work with the state-of-art in literature using linear counting algorithms and the result shows that ...
\end{abstract}

 
\section{Introduction}
\label{intro}

Traffic measurement and profiling is important in smart road systems. E.g., etc []. In this context, understanding the road conditions are important. Like finding which congestions are, by measuring metrics such as the passing vehicle numbers, delays, etc. 

One central problem is accurately knowing the profiling of traffic through vehicle to RSU communication in a privacy-aware manner, i.e., the infrastructure should not keep record of which vehicles are as they travel through the system. This is in constrast with conventional, camera based systems, which are costly, privacy-leaking, and requires accurate information of the entire system. The prohibitive cost usually prevents them from being widely used. 

This problem is important because knowing such info can help on the road network upgrades and testing. 

Existing approaches and limitations. 

In this paper, we adopt a new framework based on privacy aware approximate profiling. The insight or enabling technologies are as follows.  

Overall this is a framework with a systematic theoretical foundation.  

Assumptions: This work has assumptions, This is valid because 

Limitations: This work has limitations,  this is OK because the system does not require fully accurate. 

Its usages include: Useful as 1, 2, 3. What is the applications scenario that uses it? If it has tradeoffs (better and worse together), why it is fitting for this application scenario?
 
Key contributions: Work overview. Following is a list. 

Some details on its evaluation highlights and results.

The paper is organized as follows. 
 

















 
 

\section{Related Work}
\label{relatedwork}

In this section, we describe related work in three parts: first, second, third.

Recent progress and Comparison
 
 

\section{Design of }
\label{design}

In this section, we introduce the design of the . We first present the problem formulation. Then, we present an overview of its structure. Finally we present a detailed description of its components and related algorithms.


\subsection{The Problem Formulation}
\label{problemformulation}

The formulation of the problem is as follows: 

To be successful, it must demonstrate



\subsection{Architecture Overview}
\label{architecture}

The RSUs are installed as follows: 


The RSUs communicate with the vehicle. The steps are as follows:


Now the detailed components. Describe the motivation for design choices.  
 

The consequences of the design choices:


\subsection{Design Considerations}

Tradeoffs and pitfalls

If there are parameters, are they chosen auotmatically or self-adaptive?
 



 
\section{Analysis of }
\label{analysis}

What are the performance metrics, including the error rate, etc. Metrics definitions

What are its key performance parameters or configurations?

This system introduces the following configurable things, like parameters
 

 
\section{System Evaluation}
\label{evaluation}

\subsection{Evaluation Overview}

In this section, we systematically present the evaluation of the . Due to the , we first focus on its . Details. Then we focus on.. Details.
Define metrics
 ?


\subsection{Setup of Evaluations}

Configuration details?

Baselines?

\subsection{Evaluation Results}
 
Figures to be put here. 

 
\section{Application Case Study}
\label{casestudy}

\subsection{Problem Overview}
 

\subsection{Details}

 
 
\section{Conclusions}
\label{conclusion}
 
 



 

\end{document}
%{\small%\bibliographystyle{IEEEtran}
%\bibliography{reference}
%}

%\end{document}
