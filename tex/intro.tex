\section{Introduction}
\label{intro}

Traffic measurement and profiling is important in smart road systems. E.g., etc []. In this context, understanding the road conditions are important. Like finding which congestions are, by measuring metrics such as the passing vehicle numbers, delays, etc. 

One central problem is accurately knowing the profiling of traffic through vehicle to RSU communication in a privacy-aware manner, i.e., the infrastructure should not keep record of which vehicles are as they travel through the system. This is in constrast with conventional, camera based systems, which are costly, privacy-leaking, and requires accurate information of the entire system. The prohibitive cost usually prevents them from being widely used. 

This problem is important because knowing such info can help on the road network upgrades and testing. 

Existing approaches and limitations. 

In this paper, we adopt a new framework based on privacy aware approximate profiling. The insight or enabling technologies are as follows.  

Overall this is a framework with a systematic theoretical foundation.  

Assumptions: This work has assumptions, This is valid because 

Limitations: This work has limitations,  this is OK because the system does not require fully accurate. 

Its usages include: Useful as 1, 2, 3. What is the applications scenario that uses it? If it has tradeoffs (better and worse together), why it is fitting for this application scenario?
 
Key contributions: Work overview. Following is a list. 

Some details on its evaluation highlights and results.

The paper is organized as follows. 
 















